\documentclass[10pt,a4paper,draft]{article}
\usepackage[utf8]{inputenc}
\usepackage{amsmath}
\usepackage{amsfonts}
\usepackage{amssymb}
\usepackage{relsize}
\usepackage{mathtools}
\usepackage[final]{graphicx}
\DeclareMathOperator*{\argmin}{argmin}
\DeclareMathOperator*{\argmax}{argmax}
\newcommand*\perm[2][^n]{\prescript{#1\mkern-2.5mu}{}P_{#2}}
\usepackage{fullpage}
\usepackage{times}
\usepackage{fancyhdr}
\usepackage[ruled,vlined,linesnumbered]{algorithm2e}
\newcommand\mycommfont[1]{\footnotesize\ttfamily\textcolor{blue}{#1}}
\SetCommentSty{mycommfont}
\usepackage{xcolor}

\begin{document}
\title{RL Approach for the TSP Problem}
\author{Avrech Ben-David}
\maketitle
\begin{abstract}
In this work we review the recent works in this field. Our purpose is to point on an optional improvement direction which has not been investigated yet, and which we think has a good potential. 
\end{abstract}
\section{Introduction}
The TSP problem is a fundemental NP-hard combinatorial problem with myriad applications in the industry. Recently, many RL approaches were investigated in order to learn an efficient heuristic to solve the TSP problem. A probable reason is that many variations of the TSP problem can be simply formulated in RL terms, saving the costly development time of ad-hoc solvers. 

However, when testing the generalization ability, there is still a gap of approximately 10\% in performance between all these approaches and the best known solver \cite{concorde}, Concorde. Concorde has time complexity of $\mathcal{O}(n^22^n)$ in the worst case, but in the average case it solves instances of thousands of cities to optimality in minutes. The main weakness of Concorde is that it able to solve only the symmetric TSP. 

There is already a success in applying RL to NP-hard combinatorial problems (MVC, SCP, MAXCUT - see \cite{dai17-tsp-s2v}), but the common characteristic of all these 'solved' problems is that they are all \textit{n-choose-k}-like problems. \cite{dai17-tsp-s2v} solve MVC, SCP and MAXCUT to 1\% from the optimal, generalize well and consume much less computation resources than competitive solutions. However, they fail in the TSP. Other works, \cite{bello16-tsp-pnac} and \cite{deudon18-tsp-nr2opt}, use Pointer Networks to solve the TSP. All these works solve the instance they were trained on plausibly, but they fail to generalize to larger instances. While \cite{dai17-tsp-s2v} maintain an approximation ratio of 1.1 on a large scale of graphs (up to 1K nodes), the others fail to generalize to more than 100 nodes. 

In this work we follow \cite{dai17-tsp-s2v}, investigating their weaknesses, in order to predict a probably good improvement.

\section{Related Work}
\cite{bello16-tsp-pnac} were the train locomotive in using Pointer Networks to parameterize a stochastic policy of selecting nodes into the TSP solution. They proposed the common RL formulation for the TSP; the state is the current partial solution - an ordered list of nodes; the action is appending a node to the end of the tour; and the reward is the final tour length. They parameterized the target policy by LSTM-based pointer network, and trained it with REINFORCE-Actor-Critic, to predict a sequence of nodes one by one. They used Monte-Carlo estimator for the reward signal as it is the natural choice for the TSP, where the total reward can extremely change in the final step. Their architecture is limited to fully-connected graphs in the 2D Euclidean space, and requires a long training time. They fail in handling large instances (more than 100 nodes).

\cite{deudon18-tsp-nr2opt}, following \cite{bello16-tsp-pnac}, replaced the LSTMs with feed forward components, leaving the pointing mechanism as is. The decoder, composed of a pipeline of 3 stages, explicitely forgets after three steps. The other RL settings remained the same. Obviously, this simplified implementation achieved poor performance by its own, and for this they used 2opt, a local search algorithm to refine the solution. At the bottom line, they improved the performance of \cite{bello16-tsp-pnac} insignificantly, while reducing the model size and the training time dramatically. Anyway, they suffer from the same poor generalization, and the inference time become very long on large graphs due to the 2opt search.

A completely different approach was proposed by \cite{dai17-tsp-s2v}. They used S2V \cite{dai16-s2v} to represent the graph nodes, and learned a Q-function over these features. They trained the network end to end, via a so called n-step-DQN algorithm (see alg below). Their state is the current partial solution, the action is inserting a node to the partial solution such that the increase in the total length is minimal. The reward is the negative increase in the tour length after taking the action. Their architecture can easily handle any graph structure, and maintain performance on much larger graphs than it was trained on. 

The first and repairable problem is that the Q-function averages the nodes features, throwing away the information of the nodes order. This problem can be viewed in the order the network selects nodes to the partial solution. The network was probably planned to address the mentioned \textit{n-choose-k}-like problems, which are insensitive to the nodes prediction order.

The second and more problematic issue is the reward definition. It is not clear if even n-step reward is good enough to evaluate an action quality. As mentioned, the TSP solution cost can extremely grow in the final step, when closing the loop.

Following, in Section 3 we describe the solution in detail. First we define the RL formulation for the TSP. Next we give an introduction to the network components. The training process is explained in section 4. We detail our experimental setup and results in section 5. In section 6 we discuss our findings, and offer a future work. Conclusions are summarized in Section 7.

\section{Methodology}
\subsection{Problem Statement and RL Formulation}
The traveling salesman problem is: given a graph $G(V,E)$, find the visiting order, such that every node in $V$ is visited exactly once, and the total length of the circular tour is minimal. Representing the graph nodes as a list: $V = \{1,2,3,...,n\}$, the TSP solution $S$ is a permutation of $V$, such that:

\begin{equation} \label{tsp_statement}
	S = \argmin_{s \in Perm(V)} C(S)
\end{equation}

Where $C(S)$ is the TSP cost function - i.e. the tour length:

\begin{equation}  \label{tsp_cost}
	C(S) := \mathlarger{\sum}_{i=1}^{|S|}{distance(s_i, s_{i+1})} + distance(s_{|S|}, s_1)
\end{equation}

In the classic TSP, the $distance(\cdot,\cdot)$ can be any metric that preserve the triangle inequality. Here we deal only with the 2D Euclidean distance. 

\section{TSP RL Formulation}	
The TSP naturally lands to the RL standard formulation, mainly because the final $reward$ is well defined in the TSP statement itself. 
\cite{dai17-tsp-s2v} use DQN with \textit{experience replay} with n-step reward (see alg below). The MDP is defined as follows:
\begin{list}{}{}
	\item[•] State $s_t$ - An ordered sequence of $t$ selected nodes $\in \perm[n]{t}$, representing the current partial solution. The environment is initialized with empty solution $s_0 = \{\}$.
	\item[•] Action $a_t(v)$ - Adding a node $v \not\in S(k)$ to the partial solution, and placing it greedily to minimize the increase in the tour length $C(S)$. Adding a node inherently defines the transition from state $s_t$ to $s_{t+1}$.
	\item[•] Reward $r(s_t,a_t)$ - The change in the cost function, after taking action $a_t$ in the state $s_t$ and transitioning to state $s_{t+1}$, i.e. $C(s_{t+1})-C(s_t)$. The accumulated reward over $n$ steps $r_t^{(n)}$ is the total increase $C(s_{t+n})-C(s_t)$. The final reward of an episode is the total tour length $C(s_{|V|})$
\end{list}


\subsection{Q Network}
We follow S2V paradigm of approximating the State-Action value function.
Designing the $Q$ function is our main task. S2V-DQN uses a permutation invariant function, in the meaning that it averages the graph nodes features. Probably it was designed at first for permutation-agnostic problems, such as MVC. Our desired is a $Q$ function that is sensitive to the visiting order of the nodes, but must not overfit to specific input order. 

\cite{dai17-tsp-s2v} propose the following parameterization for $Q(s,a)$:
First, at the begining of every state, we update the S2V embeddings, by running $T$ iterations of massage passing:
\begin{equation}
	\mu_v^{(t+1)} \leftarrow Relu\Bigg(\theta_1 x_v + \theta_2 \sum_{u \in \mathcal{N}(v)} \mu_u^{(t)} + \theta_3 \sum_{u \in \mathcal{N}(v)} Relu\big(\theta_4 w(v,u)\big)\Bigg)
\end{equation}

After embedding the current state $s_t$, we use $\widehat{Q}(s,a;\theta)$ to predict the value of an optional action:
\begin{equation}
	\widehat{Q}(s,a;\theta) = \theta_5^\top Relu(concat(\theta_6 \sum_{v \in V} \mu_v^{(T)}, \theta_7 \mu_a^{(T)}))
\end{equation}

\subsection{Q Learning}
We use here a modified DQN algorithm. In the basic DQN we use the same Q architecture for the $\varepsilon-greedy$ behavior policy $\pi_b(a|s)$ and the $greedy$ target policy $\pi^*(a|s)$.

\begin{equation}
\pi^*(a|s) = \argmax_{a \notin s}{Q^*(s,a;\theta^-)}
\label{dqn-target-policy}
\end{equation}

\begin{equation}
\pi_b(a|s) = 
\begin{dcases}
    random \ {a \notin s} 					& \text{w.p  \ \ } \varepsilon \\
    \argmax_{a \notin s}{Q(s,a;\theta)} 	& \text{otherwise}
\end{dcases}
\label{dqn-behavior-policy}
\end{equation}
We use the behavior policy, parameterized by $\theta$ to select an action $a_t$ for the current state $s_t$, and the target policy, parameterized by $\theta^-$ for estimating the DQN target, originally defined as:
\begin{equation}
y_t = r_t + \gamma\max_{a'}Q(s_{t+1},a';\theta^-)
\label{orig-dqn-target}
\end{equation}
The only modification is that we use the sum of the following $n$ rewards $r_t^{(n)} = \sum_{i=0}^{n-1} r_{t+i}$, and after that bootstrap using the target policy \eqref{dqn-target-policy}:
\begin{equation}
y_t = r_t^{(n)} + \gamma\max_{a'}Q(s_{t+n},a';\theta^-)
\label{s2vdqn_target}
\end{equation}
The objective function at step $t$ is the squared error between the target and the learned $Q$ function:
\begin{equation}
L_t(\theta) = (y_t - Q(s_t,a_t;\theta)^2
\label{dqn_objective}
\end{equation}
We can now optimize this loss w.r.t $\theta$ using SGD. In order to reduce the updates variance, DQN use an \textit{experience-replay}. We retroactivelly store each state-action pair with its following accumulated rewards, and the resultant state $<s_t, a_t, r_t^{(n)}, s_{t+n}>$ in a replay memory $\mathcal{M}$. At each update we sample a batch of examples $\mathcal{B} ~^{i.i.d} \mathcal{M}$ and make a more stable update of $\theta$. The target parameters $\theta^-$ are updated every certain number of steps by $\theta$ to prevent chase after a moving target.

The learning process is detailed in algorithm \eqref{s2v-dqn-alg}.


\begin{algorithm}[H]
	\SetAlgoLined
	\DontPrintSemicolon
	\KwResult{$\theta$}
 	Initialize replay memory $\mathcal{M}$ to capacity $N$ \\
	Initialize agent network parameters $\theta$ \\
	Initialize target network parameters $\theta^- \leftarrow \theta$ \\
	Set target update period $T_{update}$ \\ 
	Set global counter $t_{global} = 1$ \\
	\For{episode 1 to L}{
		Initialize environment and state $s_0$ \\
		\For{step t = 1 to T}{
			select action $a_t$ according to \eqref{dqn-behavior-policy} \\
			observe reward $r_t$ and new state $s_{t+1}$ \\
			\If{$t \geq n$}{			
				$r_{t-n+1}^{(n)} \leftarrow \sum_{i=t-n+1}^t r_i$  \tcp*{accumulate the total reward of the last n steps} 
				store $<s_{t-n+1}, a_{t-n+1}, r_{t-n+1}^{(n)},s_{t+1}>$ in $\mathcal{M}$ \\
				sample a random mini-batch $\mathcal{B} \stackrel{i.i.d}{\sim} \mathcal{M}$ \\
				\For{$<s_j, a_j, r_j^{(n)}, s_{j+n}>$ in $\mathcal{B}$}{
					$y_j \leftarrow 
						\begin{dcases}
					    r_j^{(n)}				& \text{for terminal } s_{j+n} \\
					    r_j^{(n)} + \gamma\max_{a'}Q(s_{j+n},a';\theta^-) 	&  \text{for non-terminal } s_{j+n}
						\end{dcases} 
					$ \tcp*{calculate target}
					$L_j(\theta) \leftarrow (y_j - Q(s_j,a_j;\theta)^2$ \tcp*{calculate loss} 
				}
				perform SGD on $L_j(\theta)$ \\
			}
			\If{$t_{global} \mod T_{update} == 0$}{
				$\theta^- \leftarrow \theta$	\tcp*{update target parameters}		
			}
			$t_{global} \leftarrow t_{global} + 1$
		}
	} 
	\caption{S2V-DQN: n-step reward with Experience Replay}
 	\label{s2v-dqn-alg}
\end{algorithm}

Now we can conceptually optimize our network end-to-end.

\section{Experiments}
	The Concorde\footnote{https://github.com/jvkersch/pyconcorde} algorithm was used as the optimal solution, which the following approximation ratio and execution time ratio are related to. As a baseline model we took the cross-entropy method\footnote{https://github.com/v-iashin/CrossEntropyTSP}
\subsection{Realworld Data - Experimental Setup}
	The results below refer to the Tsplib dataset\footnote{http://dimacs.rutgers.edu/Challenges/TSP/}. The dataset contains 38 cities of 51-318 nodes each one.
	The S2V model in this experiment was trained for 100,000 epochs on a single sample (berlin52). 
	Table \ref{tb_tsplib_performance_s2v_vs_ce} shows the averaged approximation and time ratio of S2V over the entire data-set. The ratio is defined as $\dfrac{S2V_{performance}}{Concorde_{performance}} $ so as the ratio is closer to 1 the S2V works better.
	
	\begin{table}[h] \centering
	\begin{tabular}{lll}
	 	Instance Name	& S2V   		& Cross-Entropy Method 	\\
	 	berlin52 		& 1.007@1.47	& - 					\\
		eil51  			& 1.049@1.73	& 1.16@5075				\\
		st70 			& 1.065@1.56 	& 1.17@16666 			\\
		eil76			& 1.066@3.45 	& 1.12@31000
	\end{tabular}
	\caption{Tsplib Small Cities Approximation Ratio vs Time} 
	\label{tb_tsplib_performance_s2v_vs_ce}
	\end{table}
	This experiment demonstrates the ability of S2V to solve an instance as an out-of-the-box solver\footnote{This was the answer I got for why did they trained the model on a single city only.}. 
	
	Additional experiments with traditional setting are detailed below. I did not evaluated the cross-entropy method on larger graphs than 80 nodes, because it becomes extremely slow. It takes around 11 minutes to solve a graph of 52 nodes, and it seems like it grows exponentially.
	

\subsection{Synthetic Data Experiment}
In this experiment the network is trained on synthetic dataset\footnote{https://www.dropbox.com/sh/r39596h8e26nhsp/AADRm5mb82xn7h3BB4KXgETsa?dl=0}, consist of either clustered or uniformly distributed graphs.
I trained the network on 1000 graphs of 15-20 and 50-100 nodes. Table \ref{tb_tsp2d_performance_s2v} lists the achieved performance. 

\begin{table}[h] \centering
	\begin{tabular}{lll}
	 	Trained on		& Approximation @ Time Method 	\\
	 	15-20	 		& 1.092@0.91					\\
		50-100			& - 
	\end{tabular}
	\caption{Performance of S2V on TspLib for Different Train-set Size} 
	\label{tb_tsp2d_performance_s2v}
\end{table}

There is no significant improvement upon training on a single sample. It seems that this performance level really reflects the S2V architecture expressiveness.
	


\section{The Authors Explanation}
	The authors explained the fact that S2V did not show impressive performance on the TSP problem in that the tested graphs are fully-connected. The graph structure is less important, and even "graph-agnostic" methods achieve the same performance. 
	Actually, the TSP problem was not widely investigated in the paper relative to the other problems.

\section{My Conclusions}
	This work does not show outstanding performance on the TSP problem. The authors argue that it is because of the tested graphs are fully-connected, so that the 'graph structure' importance decreases, and the main advantage of the model doesn't expressed well. Such an argument should be supported in simple test. 
	
	I think that the S2V model has a built-in disadvantage regarding sequence mapping. The feedback that the model receivces from the currently constructed solution lacks the essential information about the tour path. The nodes features are summed together (in order to be invariant to permutations). It works well on MVC and probably on any other ${n \choose k}$-like problem, because the solution in such problems do not care of the nodes order. 
	In addition, some of the model's important properties are degenerated in this experiment, e.g. the n-step Q-learning. They do not give a satisfying intuition to their odd reward function (at least they say that it is a matter for future work).
	
	At the bottom-line, the S2V model, as is, do not fit the TSP problem's characteristics. 


\bibliography{tsp-rl-report}
\bibliographystyle{plain}
\end{document}